
\section{Theoretical Minimum}

\begin{answer}
	Closed systems can actually exist if the collection of objects under investigation do not interact with anything outside of the system.

	Consider the empty system $\emptyset$, an empty collection that contains no objects.
	The empty system is a closed system that actually exists.

	Implicit in establishing a closed system are the assumptions that all interactions with its objects are known and can be definitively measured.
	An open system is a system that interacts with objects outside the system.

	As a matter of logic, a system should be presumed closed and shown to be open.
	However, the failure to establish a system to be open does not suffice to definitively conclude the system is closed.
\end{answer}

\begin{answer}
	If each of the six states in the six-state system $S = \left\{1, 2, 3, 4, 5, 6\right\}$ can evolve into any of the six states in $S$, then each law is given by a function $S \xrightarrow{f} S$ that maps each state to the result of its evolution and the space of all laws that are possible for $S$ is classified by the set $C$ defined below.
	\begin{align}
		C & = \left\{ f \mid \operatorname{domain} f = \operatorname{codomain} f = S \right\}
	\end{align}
	The size $\left|C\right| = 6^6$ of $C$ is the number $6^6 = 46,656$ of such laws.
\end{answer}

\begin{answer}
	Allowable dynamical laws include that given by the following equation. 
	\begin{align}
		N(n + 1) = N(n) + 1
	\end{align}
\end{answer}

