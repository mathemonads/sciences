%!/usr/bin/env latex

\section{Quantum Mechanics}

\subsection{Systems and experiments}

% ket linearity
\begin{answer}
	\begin{enumerate}
		\item To show $\left\{\bra{A} + \bra{B}\right\} \ket{C} = \braket{A \mid C} + \braket{B \mid C}$, let $\ket{A}$ and $\ket{B}$ be ket-vectors whose bra-vectors are $\bra{A}$ and $\bra{B}$ respectively.

		      There is a ket-vector $\ket{D} = \ket{A} + \ket{B}$.
		      Its bra vector $\bra{D}$ is given by $\bra{D} = \bra{A} + \bra{B}$.
		      Interchanging bras and kets yields the following.
		      \begin{align}\label{eqn:interchange}
			      \braket{D \mid C} = \braket{C \mid D}^* = \left(\braket{C \mid A} + \braket{C \mid B}\right)^* = \braket{C \mid A}^* + \braket{C \mid B}^*
		      \end{align}
		      The final equation follows from linearity of complex conjugation.
		      Complex conjugation of each term in the right-hand expression in Equation \eqref{eqn:interchange} yields $\braket{D \mid C} = \braket{A \mid C} + \braket{B \mid C}$.
		\item To show that the complex number $\braket{A \mid A}$ is a real number for any ket-vector $\ket{A}$, recall that interchanging bras and kets yields a complex number $\braket{A \mid A} = \braket{A \mid A}^*$ that is its own complex conjugate.
		      It follows that its imaginary component is zero and thus that it is real.
	\end{enumerate}
\end{answer}

% calculation
\begin{answer}
	To show the operation given for any natural number $n$ in terms of components by
	\begin{align*}
		\braket{B \mid A} & =
		\begin{pmatrix}
			\beta_1^* & \beta_2^* & \cdots & \beta_{n-1}^* & \beta_n^*
		\end{pmatrix}
		\begin{pmatrix}
			\alpha_1     \\
			\alpha_2     \\
			\cdots       \\
			\alpha_{n-1} \\
			\alpha_n
		\end{pmatrix}                                                                                                           \\
		                  & = \beta_1^* \alpha_1 + \beta_2^* \alpha_2 + \cdots + \beta_{n-1}^* \alpha_{n-1} + \beta_n^* \alpha_n
	\end{align*}
	is an inner product, observe that it is sesquilinear
	\begin{align*}
		\bra{C} \{ \ket{A} + \ket{B} \} & = \sum_{k=1}^n \gamma_k^* \left(\alpha_k+\beta_k\right) = \sum_{i=1}^n \gamma_i^* \alpha_i + \sum_{j=1}^n \gamma_j^* \beta_j =
		\braket{C \mid A} + \braket{C \mid B}.
	\end{align*}
	and that interchanging bras and kets yields
	\begin{align}
		\braket{B \mid A} = \sum_{k=1}^n \beta_k^* \alpha_k = \left(\sum_{k=1}^n \alpha_k^* \beta_k\right)^* = \braket{A \mid B}^*
	\end{align}
	since $\left(\alpha_k^* \beta_k\right)^* = \beta_k^* \alpha_k$ holds for all $k \leq n$.

	Observe $\dsp \bra{B} \{ z \ket{A} \} = \sum_{k=1}^n \beta_k^* \left(z \alpha_k\right) = z \sum_{k=1}^n \beta_k^* \alpha_k = z \braket{B \mid A}$.
	Conjugate-linearity in the first argument follows immediately from the definition.
	Any ket-vector $\ket{A}$ satisfies $\dsp \braket{A \mid A} = \sum_{k=1}^n \alpha_k^* \alpha_k = \sum_{k=1}^n \left|\alpha_k\right|^2 \geq 0$, and the equation $\braket{A \mid A} = 0$ holds iff $\ket{A} = \ket{0}$.
\end{answer}

\subsection{Quantum state}

% orthogonality of u and d
\begin{answer}
	The inner product of the vectors $\dsp \ket{r} = \frac{1}{\sqrt{2}} \ket{u} + \frac{1}{\sqrt{2}} \ket{d}$ and $\dsp \ket{l} = \frac{1}{\sqrt{2}} \ket{u} - \frac{1}{\sqrt{2}} \ket{d}$ is $\frac{1}{2} - \frac{1}{2} = 0$, since $\ket{u}$ and $\ket{d}$ are mutually orthonormal.
	Therefore, the vector $\ket{r}$ is orthogonal to $\ket{l}$.
\end{answer}

\begin{answer}
	Let $\ket{i} = \frac{1}{\sqrt{2}} \ket{u} + \frac{i}{\sqrt{2}} \ket{d}$ and $\ket{o} = \frac{1}{\sqrt{2}} \ket{u} - \frac{i}{\sqrt{2}} \ket{d}$.
	Since $\ket{u}$ and $\ket{d}$ are mutually orthonormal, we obtain orthogonality as shown below.
	\begin{align*}
		\braket{i \mid o} & = \frac{1}{2} + \left(\frac{i}{\sqrt{2}}\right)^* \left(-\frac{i}{\sqrt{2}}\right) = 0
	\end{align*}
	By orthonormality, we also obtain the following equations.
	\begin{align*}
		\braket{o \mid u} \braket{u \mid o} & = \frac{1}{\sqrt{2}} \frac{1}{\sqrt{2}} = \frac{1}{2}                                     \\
		\braket{o \mid d} \braket{d \mid o} & = \left(-\frac{i}{\sqrt{2}}\right)^* \cdot \left(-\frac{i}{\sqrt{2}}\right) = \frac{1}{2} \\
		\braket{i \mid u} \braket{u \mid i} & = \frac{1}{\sqrt{2}} \frac{1}{\sqrt{2}} = \frac{1}{2}                                     \\
		\braket{i \mid d} \braket{d \mid i} & = \left(\frac{i}{\sqrt{2}}\right)^* \left(\frac{i}{\sqrt{2}}\right) = \frac{1}{2}
	\end{align*}
	Similarly, we obtain the following equations.
	\begin{align*}
		\braket{o \mid r} \braket{r \mid o} & = \left(\frac{1}{2} + (-\frac{i}{\sqrt{2}})^* \frac{1}{\sqrt{2}}\right)\left(\frac{1}{2} + \frac{1}{\sqrt{2}}(-\frac{i}{\sqrt{2}}) \right) = \frac{1}{4} + \frac{-i^2}{4} = \frac{1}{2} \\
		\braket{o \mid l} \braket{l \mid o} & = \left(\frac{1 + (-i)^*(-1)}{2}\right)\left(\frac{1 + (-1)(-i)}{2}\right) = \frac{1-i^2}{4} = \frac{1}{2}                                                                              \\
		\braket{i \mid r} \braket{r \mid i} & = \frac{\left(1 -i\right)\left(1 + i\right)}{4} = \frac{1 - i^2}{4} = \frac{1}{2}                                                                                                       \\
		\braket{i \mid l} \braket{l \mid i} & = \frac{\left(1 + i\right)\left(1 - i\right)}{4} = \frac{1 + 1}{4} = \frac{1}{2}
	\end{align*}
	Among pairs of orthogonal vectors $\ket{I}$ and $\ket{O}$ satisfying all of the eight equations shown above, the pair of orthogonal vectors $\ket{i}$ and $\ket{o}$ is not strictly unique.
\end{answer}

\begin{answer}
	Let $\ket{i} = \alpha \ket{u} + \beta \ket{d}$ and $\ket{o} = \gamma \ket{u} + \delta \ket{d}$ be given in terms of unknown components $\alpha$, $\beta$, $\gamma$, and $\delta$.
	\begin{enumerate}
		\item Since $\braket{u \mid i} = \alpha$ and $\braket{i \mid u} = \alpha^*$, it follows that $\alpha^* \alpha = \braket{i \mid u} \braket{u \mid i} = \frac{1}{2}$.
		      Since $\braket{d \mid i} = \beta$, it similarly follows that $\beta^* \beta = \braket{i \mid d} \braket{d \mid i} = \frac{1}{2}$.
		      Since $\braket{u \mid o} = \gamma$ and $\braket{d \mid o} = \delta$, it follows that $\gamma^* \gamma = \braket{o \mid u} \braket{u \mid o} = \frac{1}{2}$ and that $\delta^* \delta = \braket{o \mid d} \braket{d \mid o} = \frac{1}{2}$.
		\item Observe $\alpha^* \beta + \alpha \beta^* = 0$.
	\end{enumerate}
\end{answer}

