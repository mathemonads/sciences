
\section{Theoretical Minimum}

\subsection{Classical physics}

\begin{answer}
	Closed systems can actually exist if the collection of objects under investigation do not interact with anything outside of the system.

	Consider the empty system $\emptyset$, an empty collection that contains no objects.
	The empty system is a closed system that actually exists.

	Implicit in establishing a closed system are the assumptions that all interactions with its objects are known and can be definitively measured.
	An open system is a system that interacts with objects outside the system.

	As a matter of logic, a system should be presumed closed and shown to be open.
	However, the failure to establish a system to be open does not suffice to definitively conclude the system is closed.
\end{answer}

% functions of states 
\begin{answer}
	Consider a system consisting of six states $S = \left\{1, 2, 3, 4, 5, 6\right\}$.

	A law is given by a function $S \xrightarrow{f} S$ that assigns each state in $S$ to the state that results from evolution under the law.

	If any law is allowed, then the set of laws is given by the set of all functions from $S$ to $S$.

	If a law must satisfy the conservation of information, then the set of laws that are possible for a six-state system can be classified by the set of functions
	\begin{align}
		C & = \left\{ f : S \to S \mid \forall y \in S, \exists!\ x \in S. f(x) = y \right\},
	\end{align}
	where each function has the property that each element in its codomain has a corresponding, unique element in the domain that is mapped to it and thus each such function in $C$ is a bijection.
	The number of such laws, given by the size $\left|C\right| = 6!$ of $C$, is $6 \times 5 \times 4 \times 3 \times 2 \times 1 = 720$.
	This is because any function $S \xrightarrow{f} S$ in $C$ can be defined by sampling from the space $S$ of outcomes six times without replacement.
\end{answer}

\begin{answer}
	Consider the infinite system $\left\{\cdots, -1, 0, +1, +2, \cdots\right\}$.

	Examples of dynamical laws that are allowable include those given by the following equations.
	\begin{align}
		N(n+1) = N(n) - 1 \qquad
		N(n+1) = N(n) + 2 \qquad
		N(n+1) = -1^{N(n)} N(n)
	\end{align}
	On the other hand, the dynamical law given by the equation $N(n+1) = N(n)^2$ is not allowed because it does not satisfy the conservation of information.
\end{answer}

\subsubsection{Spaces}
